% Options for packages loaded elsewhere
\PassOptionsToPackage{unicode}{hyperref}
\PassOptionsToPackage{hyphens}{url}
%
\documentclass[
]{article}
\usepackage{lmodern}
\usepackage{amssymb,amsmath}
\usepackage{ifxetex,ifluatex}
\ifnum 0\ifxetex 1\fi\ifluatex 1\fi=0 % if pdftex
  \usepackage[T1]{fontenc}
  \usepackage[utf8]{inputenc}
  \usepackage{textcomp} % provide euro and other symbols
\else % if luatex or xetex
  \usepackage{unicode-math}
  \defaultfontfeatures{Scale=MatchLowercase}
  \defaultfontfeatures[\rmfamily]{Ligatures=TeX,Scale=1}
\fi
% Use upquote if available, for straight quotes in verbatim environments
\IfFileExists{upquote.sty}{\usepackage{upquote}}{}
\IfFileExists{microtype.sty}{% use microtype if available
  \usepackage[]{microtype}
  \UseMicrotypeSet[protrusion]{basicmath} % disable protrusion for tt fonts
}{}
\makeatletter
\@ifundefined{KOMAClassName}{% if non-KOMA class
  \IfFileExists{parskip.sty}{%
    \usepackage{parskip}
  }{% else
    \setlength{\parindent}{0pt}
    \setlength{\parskip}{6pt plus 2pt minus 1pt}}
}{% if KOMA class
  \KOMAoptions{parskip=half}}
\makeatother
\usepackage{xcolor}
\IfFileExists{xurl.sty}{\usepackage{xurl}}{} % add URL line breaks if available
\IfFileExists{bookmark.sty}{\usepackage{bookmark}}{\usepackage{hyperref}}
\hypersetup{
  pdftitle={lab3a\_3.2},
  hidelinks,
  pdfcreator={LaTeX via pandoc}}
\urlstyle{same} % disable monospaced font for URLs
\usepackage[margin=1in]{geometry}
\usepackage{color}
\usepackage{fancyvrb}
\newcommand{\VerbBar}{|}
\newcommand{\VERB}{\Verb[commandchars=\\\{\}]}
\DefineVerbatimEnvironment{Highlighting}{Verbatim}{commandchars=\\\{\}}
% Add ',fontsize=\small' for more characters per line
\usepackage{framed}
\definecolor{shadecolor}{RGB}{248,248,248}
\newenvironment{Shaded}{\begin{snugshade}}{\end{snugshade}}
\newcommand{\AlertTok}[1]{\textcolor[rgb]{0.94,0.16,0.16}{#1}}
\newcommand{\AnnotationTok}[1]{\textcolor[rgb]{0.56,0.35,0.01}{\textbf{\textit{#1}}}}
\newcommand{\AttributeTok}[1]{\textcolor[rgb]{0.77,0.63,0.00}{#1}}
\newcommand{\BaseNTok}[1]{\textcolor[rgb]{0.00,0.00,0.81}{#1}}
\newcommand{\BuiltInTok}[1]{#1}
\newcommand{\CharTok}[1]{\textcolor[rgb]{0.31,0.60,0.02}{#1}}
\newcommand{\CommentTok}[1]{\textcolor[rgb]{0.56,0.35,0.01}{\textit{#1}}}
\newcommand{\CommentVarTok}[1]{\textcolor[rgb]{0.56,0.35,0.01}{\textbf{\textit{#1}}}}
\newcommand{\ConstantTok}[1]{\textcolor[rgb]{0.00,0.00,0.00}{#1}}
\newcommand{\ControlFlowTok}[1]{\textcolor[rgb]{0.13,0.29,0.53}{\textbf{#1}}}
\newcommand{\DataTypeTok}[1]{\textcolor[rgb]{0.13,0.29,0.53}{#1}}
\newcommand{\DecValTok}[1]{\textcolor[rgb]{0.00,0.00,0.81}{#1}}
\newcommand{\DocumentationTok}[1]{\textcolor[rgb]{0.56,0.35,0.01}{\textbf{\textit{#1}}}}
\newcommand{\ErrorTok}[1]{\textcolor[rgb]{0.64,0.00,0.00}{\textbf{#1}}}
\newcommand{\ExtensionTok}[1]{#1}
\newcommand{\FloatTok}[1]{\textcolor[rgb]{0.00,0.00,0.81}{#1}}
\newcommand{\FunctionTok}[1]{\textcolor[rgb]{0.00,0.00,0.00}{#1}}
\newcommand{\ImportTok}[1]{#1}
\newcommand{\InformationTok}[1]{\textcolor[rgb]{0.56,0.35,0.01}{\textbf{\textit{#1}}}}
\newcommand{\KeywordTok}[1]{\textcolor[rgb]{0.13,0.29,0.53}{\textbf{#1}}}
\newcommand{\NormalTok}[1]{#1}
\newcommand{\OperatorTok}[1]{\textcolor[rgb]{0.81,0.36,0.00}{\textbf{#1}}}
\newcommand{\OtherTok}[1]{\textcolor[rgb]{0.56,0.35,0.01}{#1}}
\newcommand{\PreprocessorTok}[1]{\textcolor[rgb]{0.56,0.35,0.01}{\textit{#1}}}
\newcommand{\RegionMarkerTok}[1]{#1}
\newcommand{\SpecialCharTok}[1]{\textcolor[rgb]{0.00,0.00,0.00}{#1}}
\newcommand{\SpecialStringTok}[1]{\textcolor[rgb]{0.31,0.60,0.02}{#1}}
\newcommand{\StringTok}[1]{\textcolor[rgb]{0.31,0.60,0.02}{#1}}
\newcommand{\VariableTok}[1]{\textcolor[rgb]{0.00,0.00,0.00}{#1}}
\newcommand{\VerbatimStringTok}[1]{\textcolor[rgb]{0.31,0.60,0.02}{#1}}
\newcommand{\WarningTok}[1]{\textcolor[rgb]{0.56,0.35,0.01}{\textbf{\textit{#1}}}}
\usepackage{graphicx,grffile}
\makeatletter
\def\maxwidth{\ifdim\Gin@nat@width>\linewidth\linewidth\else\Gin@nat@width\fi}
\def\maxheight{\ifdim\Gin@nat@height>\textheight\textheight\else\Gin@nat@height\fi}
\makeatother
% Scale images if necessary, so that they will not overflow the page
% margins by default, and it is still possible to overwrite the defaults
% using explicit options in \includegraphics[width, height, ...]{}
\setkeys{Gin}{width=\maxwidth,height=\maxheight,keepaspectratio}
% Set default figure placement to htbp
\makeatletter
\def\fps@figure{htbp}
\makeatother
\setlength{\emergencystretch}{3em} % prevent overfull lines
\providecommand{\tightlist}{%
  \setlength{\itemsep}{0pt}\setlength{\parskip}{0pt}}
\setcounter{secnumdepth}{-\maxdimen} % remove section numbering

\title{lab3a\_3.2}
\author{}
\date{\vspace{-2.5em}}

\begin{document}
\maketitle

\hypertarget{import-data}{%
\subsection{Import Data}\label{import-data}}

\begin{verbatim}
## Rows: 19
## Columns: 2
## $ HairColour <chr> "LightBlond", "LightBlond", "LightBlond", "LightBlond", ...
## $ Pain       <dbl> 62, 60, 71, 55, 48, 63, 57, 52, 41, 43, 42, 50, 41, 37, ...
\end{verbatim}

\hypertarget{question-1.}{%
\subsection{Question 1.}\label{question-1.}}

Change HairColour so that the ordering is preserved from lightest to
darkest. Hint use: factor()

\begin{Shaded}
\begin{Highlighting}[]
\NormalTok{pain =}\StringTok{ }\NormalTok{pain }\OperatorTok\StringTok{ }\KeywordTok{mutate}\NormalTok{(}\DataTypeTok{HairColour =} \KeywordTok{factor}\NormalTok{(HairColour, }\DataTypeTok{levels =} \KeywordTok{c}\NormalTok{(}\StringTok{"LightBlond"}\NormalTok{, }
                                    \StringTok{"DarkBlond"}\NormalTok{, }\StringTok{"LightBrunette"}\NormalTok{, }\StringTok{"DarkBrunette"}\NormalTok{)))}
\KeywordTok{levels}\NormalTok{(pain}\OperatorTok{$}\NormalTok{HairColour)}
\end{Highlighting}
\end{Shaded}

\begin{verbatim}
## [1] "LightBlond"    "DarkBlond"     "LightBrunette" "DarkBrunette"
\end{verbatim}

\hypertarget{question-2.}{%
\subsection{Question 2.}\label{question-2.}}

Generate boxplots and QQ plots to check the ANOVA assumptions.

\begin{Shaded}
\begin{Highlighting}[]
\KeywordTok{library}\NormalTok{(ggplot2)}
\NormalTok{p1 =}\StringTok{ }\KeywordTok{ggplot}\NormalTok{(pain, }\KeywordTok{aes}\NormalTok{(}\DataTypeTok{x =}\NormalTok{ HairColour, }\DataTypeTok{y =}\NormalTok{ Pain, }\DataTypeTok{colour =}\NormalTok{ HairColour)) }\OperatorTok{+}\StringTok{ }
\StringTok{  }\KeywordTok{geom_boxplot}\NormalTok{() }\OperatorTok{+}\StringTok{ }
\StringTok{  }\KeywordTok{geom_jitter}\NormalTok{(}\DataTypeTok{width =} \FloatTok{0.1}\NormalTok{, }\DataTypeTok{size =} \FloatTok{1.5}\NormalTok{)}

\NormalTok{p2 =}\StringTok{ }\KeywordTok{ggplot}\NormalTok{(pain, }\KeywordTok{aes}\NormalTok{(}\DataTypeTok{sample =}\NormalTok{ Pain)) }\OperatorTok{+}\StringTok{ }
\StringTok{  }\KeywordTok{geom_qq}\NormalTok{() }\OperatorTok{+}\StringTok{ }\KeywordTok{geom_qq_line}\NormalTok{() }\OperatorTok{+}\StringTok{ }\KeywordTok{facet_wrap}\NormalTok{(}\OperatorTok{~}\NormalTok{HairColour) }\OperatorTok{+}\StringTok{ }\KeywordTok{theme_classic}\NormalTok{()}

\NormalTok{gridExtra}\OperatorTok{::}\KeywordTok{grid.arrange}\NormalTok{(p1, p2, }\DataTypeTok{ncol =} \DecValTok{2}\NormalTok{)}
\end{Highlighting}
\end{Shaded}

\includegraphics{lab3a_3.2_files/figure-latex/unnamed-chunk-3-1.pdf}

The variance doesn't look too equal however, it isn't conclusive because
there are so few observations. Furthermore, the QQ plots don't say much
either but there isn't strong evidence against the normality assumption
therefore, for now we assume it holds.

\hypertarget{question-3.}{%
\subsection{Question 3.}\label{question-3.}}

From looking at the boxplots it seems as if when the darkness of hair
increases the average pain decreases.

\hypertarget{question-4.}{%
\subsection{Question 4.}\label{question-4.}}

\begin{Shaded}
\begin{Highlighting}[]
\NormalTok{pain_anova =}\StringTok{ }\KeywordTok{aov}\NormalTok{(Pain}\OperatorTok{~}\NormalTok{HairColour, }\DataTypeTok{data =}\NormalTok{ pain)}
\KeywordTok{summary}\NormalTok{(pain_anova)}
\end{Highlighting}
\end{Shaded}

\begin{verbatim}
##             Df Sum Sq Mean Sq F value  Pr(>F)   
## HairColour   3   1361   453.6   6.791 0.00411 **
## Residuals   15   1002    66.8                   
## ---
## Signif. codes:  0 '***' 0.001 '**' 0.01 '*' 0.05 '.' 0.1 ' ' 1
\end{verbatim}

From the ANOVA we can see that there is at least 2 hair colour groups
whose means differ from one another.

\end{document}
